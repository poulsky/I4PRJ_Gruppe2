\chapter{Udviklingsproces}

Under udviklingen af BargainBarter projektet er der anvendt arbejdsmetoden scrum\cite{SCRUM}. Dette blev valgt da det er gruppens foretrukne arbejdsform, og fordi det passede til de opstillede læringsmål for semesterprojektet. Den anvendte udviklingsprocces er en tilpasning til den traditionelle udgave af scrum, da den originale udgave ikke passer ind i gruppens travle hverdag. De væsentligste aspekter af scrum er dog overholdt. \\ Gruppen har haft særligt fokus på følgende aspekter af scrum:

\begin{itemize}
		\item Sprint planning
		\item Sprint Retrospektiv
		\item En slags Daily Scrum
		\item Scrum Taskboard
	\end{itemize}

\noindent I de efterfølgende sektioner beskrives, hvordan de ovenstående punkter er blevet benyttet i udarbejdelsen af projektet. 

\section{Sprint planning}
Det er forsøgt at planlægge alle tasks for et sprint i starten af sprintet. Det har dog vist sig, at alle tasks ikke planlægges i starten, da de først opstår undervejs i sprintet. Dette ser gruppen ikke som noget problem, så længe at scrum boardet bliver holdt opdateret med de nye opgaver. Det er anden gang, at vi har arbejdet med scrum. Og på baggrund af tidligere opnåede erfaringer har vi forsøgt at sørge for, at tasks der bliver lagt ind på scrum boardet skal være konkrete og små. Dette gør opgaven mere overskuelig og attraktiv for gruppens medlemmer.  


\section{Sprint Retrospektiv}
Sprint Retrospektivet har for vores samarbejde været af særlig værdi. Dette skyldes at retrospektivet er tidspunktet hvor vi udvikler og tilpasser vores proces Vi får strammet op på de processer, der ikke fungerer. Selve retrospektivet foregår ved, at vi gennemgår hver især hvad vi synes der har fungeret hhv. godt og skidt i processen i det pågældende sprint. Derefter laves der en opsamling på de nævnte punkter.\\ Et konkret eksempel på, hvad det er blevet nævnt et sprint retrospektiv:

\begin{itemize}
	\item Der skulle være noget mere kode  i sprintet
	\item Research task skulle være mere konkrete
	\item Vores arbejde skal være konkrete ting, det er det vi lærer af
\end{itemize}

\noindent Dette er blot et eksempel på, hvad der er blevet nævnt under retrospektivet. Retrospektivet har været et afgørende punkt for at øge vores effektivitet

\section{Daily Scrum}
I projektet har vi brugt en modificeret version af daily scrum. Vi synes ikke, det gav mening at afholde scrum møder hver dag, når arbejdet ikke foregik på daglig basis. I stedet valgte vi faste dage, hvor vi afholdte stå-op møder. Til stå-op møderne fortalte vi hver især, hvad vi var i gang med, hvordan det gik, og om vi havde nogle aktuelle udfordringer. Stå-op møderne gav mulighed for, at vi som gruppe fik overblik, og at vi kunne lave fælles problemløsning på de individuelle udfordringer. Denne fælles gruppe løsning var meget gavnlig for gruppen, da vi kunne trække på fælles erfaring og viden.

\section{Gruppens process værktøjer}
Til alle Scrum relaterede aktiviteter har vi anvedt scrumwise \cite{SCRUMWISE} og til samling og versionstyring af filer og kode er der anvendt Git\cite{GIT}. 


