\chapter{Udviklingsproces}

Under udviklingen af Bargain Barter projektet er der anvendt arbejdsmetoden Scrum. Dette blev valgt da det er gruppens foretrukne arbejdsform, og som fordi det passede til de opstillede læringsmål for semesterprojektet. Den brugte udviklingsprocces er en tilpasning til den traditionelle model, da gruppen derigennem har opnået bedst performance. De væsentligste aspekter af scrum er dog opretholdt hvor gruppen har haft særligt fokus på:

\begin{itemize}
		\item Sprint planning
		\item Sprint Retrospektiv
		\item En slags Daily Scrum
		\item Scrum Taskboard
	\end{itemize}

I de efterfølgende sektioner beskrives de ovenstående punkter. Det er valgt at gøre i personligt sprog da punkterne omhandler personalrelationer og ikke er af teknisk karakter. 

\section{Sprint planning}
Det er forsøgt at gruppere indsættelsen af tasks i starten af hvert sprint. Det har dog vist sig at der er nogle tasks der ikke kan planlægges da de først opstår undervejs. Dette ser gruppen heller ikke som noget problem så længe man er opmærksom på fast at opdatere scrum boardet. Det er anden gang at vi har arbejdet med scrum, og fra tidligere opnåede erfaringer har vi forsøgt at sørge for tasks der bliver lagt ind på boardet skal være ret konkrete, og helst rimeligt små, da det gør det nemmere at tage opgaven på sig.  


\section{Sprint Retrospektiv}
Sprint Retrospektivet har for vores samarbejde været af særlig værdi. Dette skyldes at retrospektivet er tidspunktet hvor vi udvikler på vores procces, og får strammet op på de procceser der ikke fungerer. Selve mødet foregår ved at vi gennemgår hver især hvad vi synes der går hhv. godt og skidt i processen i projektet. Derefter laves der en opsamling, og et konkret eksempel på hvad det fx kunne være der blev nævnt er:

\begin{itemize}
	\item Der skulle være noget mere kode  i sprintet
	\item Research task skulle være mere konkrete
	\item Vores arbejde skal være konkrete ting, det er det vi lærer af
\end{itemize}

Dette er blot eksempel på hvad der er blevet nævnt under retrospektivet. Retrospektivet er været et afgørende punkt for at øge vores arbejdseffiktivitet.

\section{Daily Scrum}
I projektet har vi brugt en version af daily scrum, men ikke på helt traditionel vis. Vi synes ikke det gav mening at afholde scrum møder hver dag, når arbejdet ikke foregik på daglig basis. Istedet valgte vi faste dage hvor vi holdte stå op møder, og hvor vi hver især fotalte hvad vi var igang med, hvordan det gik og om vi havde nogle aktuelle udfordringer. Stå op møderne gav mulighed for at vi som gruppe fik overblik, og at vi kunne lave problemløsning på individuelle udfordringer, hvilket var meget gavnligt

\section{Gruppens process værktøjer}
Til alle Scrum relaterede aktiviteter har vi anvedt scrumwise og til samling og versionstyring af filer og kode er der anvendt .git. 


