\chapter{Samarbejde}

Under udviklingen af Bargain Barter projektet er der gjort brug af arbejdsmetoden Scrum. Dette var den af gruppen foretrukne arbejdsform, som også gav mening ift. de opstillede læringsmål for projektet. Den brugte udviklingsprocces er ikke direkte scrum, men istedet en tilpasset afart da gruppen har valgt kun at bruge de dele af scrum der bringer værdi til gruppen. De væsentligste aspekter af scrum er til gengæld opretholdt hvor gruppen har haft særligt fokus på:

\begin{itemize}
	\item Sprint planning
	\item Sprint Retrospektiv
	\item En slags Daily Scrum
	\item Scrum Taskboard
\end{itemize}

I de efterfølgende sektioner beskrives disse. Det er valgt at gøre i personligt sprog fordi det virkede mere passende til denne type sektioner. 

\section{Sprint planning}
Det er forsøgt at gruppe indsættelsen af tasks i starten af hvert sprint, det har dog vist sig at der er nogle tasks der ikke kan planlægges da de først opstår undervejs. Dette ser gruppen heller ikke som noget problem så længe man er opmærksom på fast at opdatere scrum boardet.  


\section{Sprint Retrospektiv}
Sprint Retrospektivet har for vores samarbejde været af kæmpe værdi. Det er i retrospektivet hvor vi har haft mulighed for at udvikle på vores procces, og få strammet op på de procceser der ikke fungere. Der er flere eksempler på ting der er blevet gennemtrumpfet under retrospektiv, men det valgte eksempel omhandler task boardet. \\

Scrum boarded var i perioder ikke udfyldt i tilpas grad, og der blev vi som gruppe enige om at vi mistede overblikket over projektet hvis vi ikke kunne se hvad folk arbejdede med, og vi fik mulighed for at italesætte at det var et problem, og derefter blev vi meget bedre til at opdatere scrumboardet og blev også mere effektive af det.

\section{Daily Scrum}
IKKE FÆRDIGT!

I projektet har vi brugt en version af daily scrum, men ikke på helt traditionel vis. Vi har ikke synes det gav mening at afholde scrum møder hver dag, men istedet har vi valgt nogle faste dage hvor vi har haft mulighed for lige at sige hvad vi var igang med og mindst lige sige hvordan det går, og om vi havde nogle udfordringer. Det vigtigste for gruppen har i dette tilfælde været muligheden for at kunne beskrive hvis der var noget der stod i vejen for en. 

\section{Scrum TaskBoard}

\section{Generelt om samarbejde}
