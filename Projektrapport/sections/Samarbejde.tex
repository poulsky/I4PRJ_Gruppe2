\chapter{Udviklingsproces}

Under udviklingen af BargainBarter projektet er der anvendt arbejdsmetoden scrum\cite{SCRUM}. Dette blev valgt da det er gruppens foretrukne arbejdsform, og fordi det passede til de opstillede læringsmål for semesterprojektet. Den anvendte udviklingsproces er en tilpasning til den traditionelle udgave af scrum, da den originale udgave ikke passer ind i gruppens travle hverdag. De væsentligste aspekter af scrum er dog overholdt. \\ Gruppen har haft særligt fokus på følgende aspekter af scrum:

\begin{itemize}
		\item Sprint planning
		\item Sprint Retrospektiv
		\item Daily Scrum
		\item Scrum Taskboard
	\end{itemize}

\noindent I de efterfølgende sektioner beskrives, hvordan de ovenstående punkter er blevet benyttet i udarbejdelsen af projektet. 

\section{Sprint planning}
Vi bestræbede os på at planlægge så mange tasks som muligt inden begyndelsen af et sprint. Det var dog ikke muligt at undgå, at der ikke opstod nye tasks undervejs i sprintet. Dette ser vi ikke som noget problem, så længe at scrum boardet bliver holdt opdateret med de nye opgaver. På baggrund af tidligere opnåede erfaringer med scrum, har vi forsøgt at sørge for, at tasks der bliver lagt ind på scrum boardet skal være konkrete og små. Dette gør opgaven utvetydig og overskuelig for gruppens medlemmer.   


\section{Sprint Retrospektiv}
Sprint Retrospektivet har været et vigtigt led i arbejdsprocessen. Dette skyldes at retrospektivet er tidspunktet hvor vi udvikler og tilpasser vores proces. Der bliver optimeret på de processer, der ikke fungerer, og man sikrer at god praksis vedligeholdes. Selve retrospektivet foregår ved, at vi hver især gennemgår hvad der har fungeret hhv. godt og skidt i processen i det pågældende sprint. Et konkret eksempel på, hvad det er blevet nævnt et sprint retrospektiv:

\begin{itemize}
	\item Der skulle være kodet mere i sprintet
	\item Research tasks skulle være mere konkrete
	\item Vi skal være bedre til at møde til tiden
\end{itemize}


\section{Daily Scrum}
I projektet har vi brugt en modificeret version af daily scrum. Vi synes ikke, det gav mening at afholde scrum møder hver dag, når arbejdet ikke foregik på daglig basis. I stedet valgte vi faste dage, hvor vi afholdte stå-op møder. Til stå-op møderne fortalte vi hver især, hvad vi var i gang med, hvordan det gik, og om vi havde nogle aktuelle udfordringer. Stå-op møderne gav mulighed for, at vi som gruppe fik overblik, og at vi kunne lave fælles problemløsning på de individuelle udfordringer.

\section{Procesværktøjer}
Til alle Scrum relaterede aktiviteter har vi anvedt scrumwise\cite{SCRUMWISE} og til samling og versionstyring af filer og kode er der anvendt Git\cite{GIT}. 


