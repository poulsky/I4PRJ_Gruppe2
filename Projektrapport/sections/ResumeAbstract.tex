\chapter{Resume og Abstract}

\section{Resume}
I denne rapport beskrives udarbejdelsen af webapplikationen BargainBarter som semesterprojekt på fjerde semester IKT på IHA. Formålet med dette projekt er at udvikle en online platform, som tillader valutaløse byttehandler mellem brugere, som er tæt på hinanden.
\\ Systemet er blevet realiseret som en webapplikation udviklet med ASP.Net MVC. Systemet lader brugere oprette og søge efter bytteannonver. Vha. Google Maps API kan brugeren filtrere sin søgning efter fysisk afstand til annoncen. Systemet implementerer et ratingsystem, som tillader brugere at give hinanden en rating efter en afsluttet byttehandel. Denne rating bidrager til brugerens gennemsnitlige rating, som bliver vist sammen med bytteannoncerne på siden, således at brugerne har bedre mulighed for at vurdere hinandens troværdighed. Det er muligt for brugere at kommunikere, både på annoncesidernes kommentarsektion, og via chat-funktionen lavet med SignalR.

%\\Rapporten indeholder bl.a. information om udviklingsmetoder, design overvejelser og nok information til at en læser med samme tekniske baggrund som gruppen vil kunne danne sig en overordnet forståelse for systemets struktur og udarbejdelse. 
%\\I rapporten findes der desuden relevante referencer til dokumentation i tilfælde af, at læser ønsker mere information om en given funktionen i systemet. 
%\\Systemet opfylder "Must have" og "Should have" kravene fra MoSCoW analysen. Det betyder at det færdige produkt i denne iteration har mange funktionaliteter, men har rig mulighed for videreudvikling. @

\section{Abstract}
This report describes the development of the webapplication BargainBarter on the fourth semester in the study field IKT at IHA. The goal for this project is to develop an online platform that allows users in close proximity to trade any sort of item with eachother without the use of currency. 
\\This has been achieved by developing a webapplication using ASP.Net MVC. The system lets users create ads for trade, as well as search for other ads. Via use of Google Maps API it is possible to search for ads sorted by proximity to the user. Using the rating system it is possible to rate other users after completing a trade. The ratings will add to a user's average rating, which will be shown along with his/her ads, allowing users to better judge the credibility of a given user. It is possible for users to communicate with both the comment section on an ad, and with the chat feature using SignalR.



