\chapter{Opnåede Erfaringer}

Igennem projektet har vi lært meget om tilgangen til helt uberørte teknologier. Gruppen havde absolut ingen erfaring med webudvikling, ASP.NET og databaser, hvilket gav en del udfordringer i starten af projektet. Der blev brugt en del tid på at gennem gå tutorials og læse omkring teknologierne. Men vi startede hurtigt med at begynde på at prøve ting af, da vi mente det ville give den bedste indlæring og fornemmelse af de nye teknologier. Dette viste sig at være en god strategi. Derfor vil denne strategi også anvendt i fremtidige projekter, hvor der skal benyttes ukendt teknologi. Konklusion var, at det i bund og grund bare handler om at få prøvet nogle ting af.\\


Projektet har givet os bedre indsigt i den iterative process, og vigtigheden af versionstyrings værktøjer som git. Hele projektet er blevet kørt i iterationer, som på ingen måde var nyt for gruppen, men da der er blevet arbejdet med lokale databaser og et fælles ASP.NET projekt, har rettelser på blandt de forskellige gruppemedlemmer flere gange givet problemer. Hvis gruppen havde været foruden git, var disse ændringer blot forsvundet, men pga. versionstyringsværktøjet har de forskellige ændringer kunne findes og flettes på en måde hvor intet er gået tabt for gruppen. Dette er især brugbart når teamet udvikler på så forskellige tidspunkter, hvor man hele tiden skal være opdateret med nyeste ændringer.

Projektet er blevet udarbejdet iterativt med SCRUM som den anvendte udviklingsproces. Hvilket har givet gruppen en større forståelse for brugen af scrum og den attributer f.eks. Task-boardet. Vedrørende task-boardet var den vigtigste erfaringer, at task skulle være veldefinerede, så alle kunne påtage sig opgaverne. På task-boardet blev det erfaret, at der aldrig skulle være mangel på opgaver, da mangel på opgaver på Task-boardet var den største flaskehals i forbindelse med udarbejdelsen af projektet.\\

Igennem testningen af projektet har vi fået noget erfaring i hvad man er i stand til at teste i forbindelse med webudvikling. Det var meget kryptisk til at starte med, men vi lærte hvad der gav mening at teste, og vi lærte også at vores designmæssige overvejser i høj grad havde indflydelse på testene. Fx var repository pattern afgørende for selve testbarheden af systemet, da en database er meget svær at mocke ud i unittest, hvis der ikke ligger et ekstra lag i mellem databasen og controllerne.

\noindent Projektet har givet os bedre indsigt i den iterative process, og vigtigheden af versionstyrings værktøjer som git. Hele projektet er blevet kørt i iterationer, som på ingen måde var nyt for gruppen, men da der er blevet arbejdet med lokale databaser og et fælles ASP.NET projekt, har rettelser på blandt de forskellige gruppemedlemmer flere gange givet problemer. Hvis gruppen havde været foruden git, var disse ændringer blot forsvundet, men pga. versionstyringsværktøjet har de forskellige ændringer kunne findes og flettes på en måde hvor intet er gået tabt for gruppen. Dette er især brugbart når teamet udvikler på så forskellige tidspunkter, hvor man hele tiden skal være opdateret med nyeste ændringer.

Projektet er blevet udarbejdet iterativt med SCRUM som den anvendte udviklingsproces. Hvilket har givet gruppen en større forståelse for brugen af scrum og den attributer f.eks. Task-boardet. Vedrørende task-boardet var den vigtigste erfaringer, at task skulle være veldefinerede, så alle kunne påtage sig opgaverne. På task-boardet blev det erfaret, at der aldrig skulle være mangel på opgaver, da mangel på opgaver på Task-boardet var den største flaskehals i forbindelse med udarbejdelsen af projektet.\\

\noindent Igennem testningen af projektet har vi fået noget erfaring i hvad man er i stand til at teste i forbindelse med webudvikling. Det var meget kryptisk til at starte med, men vi lærte hvad der gav mening at teste, og vi lærte også at vores designmæssige overvejser i høj grad havde indflydelse på testene. Fx var repository pattern afgørende for selve testbarheden af systemet, da en database er meget svær at mocke ud i unittest, hvis der ikke ligger et ekstra lag i mellem databasen og controllerne.


Vi har også opnået en større erfaring med anvendelse af CI i forbindelse med udarbejdelsen af et projekt. Som CI-værktøj er TeamCity bevist blevet anvendt for at opnå erfaring med et andet CI-værktøj end Jenkins, der kendes fra I4SWT.


