\chapter{Opnåede Erfaringer}

Igennem projektet har vi lært meget om tilgangen til helt uberørte teknologier. Gruppen havde absolut ingen erfaring med webudvikling, ASP.NET og databaser, hvilket gav en del udfordringer i starten af projektet. Der blev brugt en del tid på at gennemgå tutorials og læse omkring teknologierne. Vi startede hurtigt med at prøve ting af, da vi mente det ville give den bedste indlæring og fornemmelse af de nye teknologier. Dette viste sig at være en god strategi. Derfor vil denne strategi også blive anvendt i fremtidige projekter, hvor der skal benyttes ukendt teknologi. Konklusion var, at det i bund og grund handler om at få prøvet nogle ting af.\\


Projektet har givet os bedre indsigt i den iterative process, og vigtigheden af versionstyrings værktøjer som git. Hele projektet er blevet kørt i iterationer, som på ingen måde var nyt for gruppen, men da der er blevet arbejdet med lokale databaser og et fælles ASP.NET projekt, har rettelser blandt de forskellige gruppemedlemmer flere gange givet problemer. Hvis gruppen havde været foruden git, var disse ændringer blot forsvundet, men pga. versionstyringsværktøjet har de forskellige ændringer kunne findes og flettes på en måde hvor intet er gået tabt for gruppen. Dette er især brugbart når teamet udvikler på forskellige tidspunkter, hvor man hele tiden skal være opdateret med de nyeste ændringer.

Projektet er blevet udarbejdet iterativt med SCRUM som anvendt udviklingsproces. Gruppen har derfor fået en større forståelse for brugen af scrum og dets attributer f.eks. Task-boardet. Vi kunne fra Task-boardet tage med at tasks skal være veldefinerede, så alle kunne påtage sig opgaverne. På task-boardet blev det erfaret, at der ikke må være mangel på opgaver, da dette var den største flaskehals i forbindelse med udarbejdelsen af projektet.\\

Igennem testningen af projektet har vi fået noget erfaring i hvad man er i stand til at teste i forbindelse med webudvikling. Det var meget kryptisk til at starte med, men vi lærte hvad der gav mening at teste, og vi lærte også at vores designmæssige overvejelser i høj grad havde indflydelse på testene. Repository pattern er f.eks. afgørende for selve testbarheden af systemet, da en database er meget svær at mocke ud i en unittest, hvis der ikke ligger et ekstra lag i mellem databasen og controllerne.


Vi har også opnået en større erfaring med anvendelse af CI i forbindelse med udarbejdelsen af et projekt. CI-værktøjet TeamCity blev bevidst anvendt for at opnå erfaring med et andet CI-værktøj end Jenkins, der kendes fra I4SWT.


