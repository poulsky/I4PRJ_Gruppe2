\chapter{Opnåede Erfaringer}

Igennem projektet har vi lært meget om tilgangen til hidtil ukendte teknologier. Gruppen havde ingen erfaring med webudvikling, ASP.NET og databaser, hvilket præsenterede store og små udfordringer i starten af projektet. Gruppen brugte derfor en væsentlig portion tid på at gennemgå tutorials, og læse omkring teknologierne. Vi startede tidligt i forløbet med at prøve ting af, da vi mente det ville give den bedste indlæring og fornemmelse af de nye teknologier. Dette viste sig at være en god strategi. Derfor vil denne strategi også blive anvendt i fremtidige projekter, hvor der skal benyttes ukendt teknologi. Konklusion var, at det i bund og grund handler om at få prøvet nogle ting af.\\


Projektet har givet os bedre indsigt i den iterative process, og vigtigheden af versionsstyrings værktøjer som Git. Projektet er blevet kørt i iterationer, som ikke var nyt for gruppen, men arbejdet med lokale databaser og et fælles ASP.NET projekt, har betydet at rettelser blandt de forskellige gruppemedlemmer flere gange givet problemer. Hvis gruppen havde været foruden Git, var disse ændringer blot forsvundet, men pga. versionsstyringsværktøjet har de forskellige ændringer kunne findes og flettes ind på en måde, hvor intet er gået tabt for gruppen. \\

Projektet er udarbejdet iterativt med scrum. Gruppen har derfor fået større forståelse for brugen af scrum, og dets attributter f.eks. Task-boardet. Vi lærte at tasks skal være veldefinerede, så alle kunne påtage sig opgaverne, ud fra beskrivelsen. Desuden erfarede vi, at der ikke må være mangel på opgaver, da dette var den største flaskehals i forbindelse med udarbejdelsen af projektet. Flaskehalsen var her, at gruppemedlemmer sad uden noget at lave, fordi der manglede generelt overblik.\\

Igennem testningen af projektet har vi fået erfaring i, hvad man er i stand til at teste i forbindelse med webudvikling. I starten var det meget kryptisk, men vi lærte hvad der gav mening at teste, og vi lærte at vores designmæssige overvejelser i høj grad havde indflydelse på testene. Repository pattern er f.eks. afgørende for selve testbarheden af systemet, da en database er svær at mocke ud i en unittest, hvis der ikke ligger et ekstra lag i mellem databasen og controllerne.\\

Vi har desuden opnået en større erfaring med anvendelse af CI i forbindelse med udarbejdelsen af et projekt. CI-værktøjet TeamCity blev bevidst anvendt for at opnå erfaring med et andet CI-værktøj end Jenkins, der kendes fra I4SWT.


