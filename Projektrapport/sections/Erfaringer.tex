\chapter{Opnåede Erfaringer}

Igennem projektet har vi lært meget om tilgangen til helt uberørte teknologier. Gruppen havde absolut ingen erfaring med webudvikling, ASP.net og databaser, hvilket gav udfordringer i starten af projektet. Der blev brugt en del tid på at gå tutorials igennem, men vi forsøgte huritigt at begynde på at prøve ting af, da vi mente det ville give den bedste indlæring. Det lykkedes godt, og fremtidigt vil vi sætte os ind i nye teknologier på denne måde. Det handler i bund og grund bare om at få prøvet nogle ting af.  

Projektet har givet os bedre indsigt i den iterative process, og vigtigheden af versionstyrings værktøjer. Hele projektet er kørt i iterationer, som på ingen måde var nyt for gruppen, men da der er blevet arbejdet med lokale databaser og et fælles ASP.net projekt, har rettelser blandt de forskellige brugere flere gange givet problemer. Hvis gruppen havde været foruden git, var disse ændringer blot forsvundet, men pga. versionstyringsværktøjet har de forskellige ændringer kunne findes og flettes på en måde hvor intet er gået tabt for gruppen. Dette er især brugbart når teamet udvikler på så forskellige tidspunkter, hvor man hele tiden skal være opdateret med nyeste ændringer.

Igennem testning af projektet har vi fået noget erfaring i hvad man er i stand til at teste i forbindelse med webudvikling. Det var meget kryptisk til at starte med, men vi lærte hvad der gav mening at teste, og vi lærte også at vores designmæssige overvejser i høj grad havde indflydelse på testene. Fx var repository pattern afgørende for selve testbarheden.

Vi var gode til scrum. 
