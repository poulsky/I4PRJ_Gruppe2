\chapter{Kravspecifikation}

I projektet er der valgt at benytte user stories til at beskrive den ønskede funktionalitet. Disse stories er lavet på baggrund af MoSCoW analysen.
De enkelte user stories er blevet samlet til en række af epics, der viser systemets overordnede funktionalitet. Disse epics vises på aktør-kontekst diagrammet, der ses på figur. \ref{fig:KontekstDia}. 

\begin{figure}[H]
	\includegraphics[width=140mm,height=140mm]{../Dokumentation/figures/KontekstDiagram.PDF}
	\caption{Aktør-kontekst diagram for BargainBarter}
	\label{fig:KontekstDia}
\end{figure}

Igennem rapporten vil der bliver fokuseret på følgende to user stories: "Oprette en bytteannonce" og "Søg efter bytteannonce". Dette er valgt for at kunne gå i dybden med disse user stories med hensyn til deres arkitektur, design og implementering. Disse user stories er valgt, da det er de to mest centrale funktionaliter for systemet.For en fuld liste af User stroies henvises til dokumentationen \footnote{Se bilag - Dokumentation, sektion XX}\\
De to user stories kan læses nedenunder:

\section{Oprette en bytteannonce}
{\color{blue}\textbf{EGENSKAB}:} Oprette en bytteannonce \\
Som bruger \\
Ønsker jeg at kunne oprette en bytteannonce \\
For at kunne bytte med andre brugere af systemet.\\ \\
{\color{blue}\textbf{BAGGRUND}} \\
{\color{blue}\textbf{Givet}} at bruger er logget ind \\

{\color{blue}\textbf{SCENARIE:}} Oprette en bytteannonce \\
{\color{blue}\textbf{Når}}  bruger ønsker at oprette en bytteannonce i systemet \\
{\color{blue}\textbf{Så}} navigerer han til menupunktet ”Opret annonce” \\
{\color{blue}\textbf{Så}} udfylder han bytteannonce-skabelonen \\
{\color{blue}\textbf{Og}} trykker på ”Opret annonce”-knappen
\section{Søg efter bytteannoncer}
{\color{blue}\textbf{EGENSKAB}:}Søg efter bytteannoncer \\
Som bruger \\
Ønsker jeg at kunne søge efter bytteannoncer \\
For at kunne finde en bestemt type vare\\ \\
{\color{blue}\textbf{BAGGRUND}} \\
{\color{blue}\textbf{Givet}} at bruger er logget ind \\
\\
{\color{blue}\textbf{SCENARIE:}} Søg efter bytteannoncer \\
{\color{blue}\textbf{Når}}bruger ønsker at søge efter bytteannoncer i systemet\\
{\color{blue}\textbf{Så}} navigerer han til menupunktet ”Søg” \\
{\color{blue}\textbf{Så}} indtaster han søgekriterier\\
{\color{blue}\textbf{Og}} trykker på “søg“-knappen

\section{Ikke-Funktionelle krav}
I udarbejdelsen af systemet er der blevet fastsat nogle ikke-funktionelle krav til systemet.
\chapter{Ikke-funktionelle krav}

\begin{enumerate}
	\item Man skal kunne komme ind på alle annoncer med kun 2 klik fra forsiden (link til hjemmeside)
	
	\item Hjemmesiden skal kunne håndtere 10 brugere på samme tid
	
	\item Ved brug af søgefeltet under ''annonce''-siden skal der maksimalt gå 15 sekunder fra man trykker på ''søg''-knappen før resultaterne er fundet frem og er blevet vist på skærmen
	
%	\item To personer skal kunne oprette en annonce på hver deres computer samtidig, uden at disse påvirker hinanden
	
	\item Hvis en bruger søger efter en annonceplacering inden for en vis afstand, må den maksimale afstand til den pågældende annonceplacering højst overskride den valgte afstand med 100 meter
	
	\item Hvis bruger A ønsker at kontakte bruger B gennem den indbyggede chat, så skal der ikke gå længere end 10 sekunder fra bruger A trykker til ''send'' til at beskeden afleveret til bruger B
	
	\item Når en bruger ønsker at vurdere en anden bruger, så skal man maksimalt kunne skrive 500 tegn
	
	\item Når en bruger kommenterer en anden brugers annonce, skal der maksimalt gå 15 sekunder fra, at den kommenterende bruger trykker ''Send'' til at den anden bruger kan se kommentaren på den pågældende annonce
	
	\item System skal ligge online (fast IP)
	
\end{enumerate}