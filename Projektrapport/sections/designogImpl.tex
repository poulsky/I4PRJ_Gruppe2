\chapter{Design}
I projektet har der været fokus på at implementere hele features, og ikke enkelte lag af gangen. I en lagdelt arkitektur har det betydet at alle lag er blevet indraget i implementeringen af en given feature. I dette afsnit er det valgt at beskrive to features, for at eksemplificere selve designet af systemet. Disse er \textit{Oprette en bytteannonce} og \textit{Søg efter bytteannoncer}. For fuld beskrivelse af systemdesignet, se dokumentationen \footnote{Se bilag - Dokumentation, sektion 9}.

\section{Modeldesign}

Ud fra domæneanalysen er data der giver mening at persistere fundet, og på fig. \ref{fig:BarteradModel} A kan det ses hvad der fra domæneanalysen skulle gemmes i forbindelse med Barterads.

\begin{figure}[H]
	\centering
	\includegraphics
	[width=140mm]{figures/BarterAdModels.pdf}
	\caption{A) BarterAd model fra domæneanalyse  B) Aktuel model}
	\label{fig:BarteradModel}
\end{figure} 

\noindent Der er blevet arbejdet agilt på projektet, og efter flere iterationer har designet ændret sig. På figur \ref{fig:BarteradModel} B ses det endeligt udfærdiget modeldesign af BarterAds.
\\ \\ 
\noindent I designfasen blev det bestemt hvilke sikkerhedskriterier, data i databasen skal overholde. Til dette konkrete eksempel er der en række krav til modellen som skal være opfyldt. Disse ses her:

\begin{itemize}
	\item Barterads skal være tilknyttet en ApplicationUser
	\item Barterads skal have et oprettelsestidspunkt, men dette laves her
	\item Barterads må have en maksimal beskrivelseslængde på 500 tegn
\end{itemize}

\noindent Modellen opretter igennem Entity Framework databasestrukturen, og derefter tilgås dataene vha. et repository pattern\footnote{Se bilag - Dokumentation, sektion 9.8}, der standardiserer måden at tilgå data på.


\section{Controller}  

I controllerne ligger selve funktionaliteten af BargainBarter systemet. Det er controllernes opgave at opdatere det view, som brugeren ser, samt styre kommunikationen mellem viewet og modellen. \\
 
\subsection{Oprette en bytteannonce}
\noindent Som det ses på figur \ref{fig:SDOpretBarterAd} trykker brugeren ind for at lave en ny annonce. Systemet registrerer at der er trykket på en knap, og kalder den tilhørende action. Igennem denne action returneres \textit{Create BarterAd} viewet. I dette view kan der indtastes data til Barterads. Brugeren indtaster data og trykker \textit{Opret}, hvorved controlleren opretter og persisterer Barterad'en udfra de indtastede data.  \\


\begin{figure}[H]
	\centering
	\includegraphics
	[width=140mm]{../Dokumentation/figures/SDOpretBytteAnnonce.PDF}
	\caption{Sekvensdiagram for oprettelse af barterads}
	\label{fig:SDOpretBarterAd}
\end{figure}

\noindent Beskrivelse af anonnceoprettelse er en "happy path" beskrivelse, som ikke tager højde for fejl. I design og implementering af alle controllers skal der holdes styr på mulige fejl. For eksempel er der i oprettelsen sikret at brugeren skal være logget ind. Dette holder controlleren styr på. Desuden tjekker viewet og modellen at alle datafelter er udfyldte, og fortæller brugeren hvad de mangler at udfylde.

\subsection{Søg efter bytteannoncer}     
Controllerfunktionaliteten er også valgt at eksemplificere med søgning af Barterads. Denne er taget med da det er en essentiel feature på hjemmesiden.

På hjemmesiden kan der søges efter annoncer på flere måder. Det kan være på kategori, titel og på afstand. I dette eksempel uddybes søgning efter annoncer filtreret efter afstand. Desuden eksemplificerer implementeringen af denne feature den generelle virkemåde af samspillet mellem lagene i systemet.  \\ \\

\subsubsection{Valg af løsning}
Der blev i forundersøgelsen til denne feature overvejet flere løsningsmuligheder. Der var mulighed for at få eksempelvis Google Maps API til at lave en ruteplan, og ud fra den bestemme afstanden til andre brugere. Denne løsning var ikke optimal af flere årsager. For det første kunne det blive meget performance-tungt at lave alle de forespørgsler til Google Maps API. For det andet kunne der let opstå begrænsninger ift. de maksimale antal forespørgsler som Google Maps API tillader. For det tredje blev det vurderet, at det ikke var nødvendigt med meget høj præcision på afstanden til en annonces ejer.  \\ \\

\noindent En anden løsning der var på tale, var blot at sortere efter postnumre. Dette blev dog vurderet til at være for upræcist, da arealet af et postnummer kan være meget stort. 
\\ \\

\noindent Løsningen som blev udtænkt baserer sig på beregning af afstand i fugleflugt mellem koordinater. Ved oprettelse af en brugerprofil indtaster brugeren en adresse. Vha. Google Maps API udtrækkes koordinater baseret på adressen, og disse koordinater gemmes som attributter på modellen for en bruger. Selve beregningen af afstand fungerer som en simpel trigonometrisk udregning. Denne løsning er et godt kompromis mellem høj præcision, performance og overskuelig implementering.

\subsubsection{Implementering}
\noindent Selve søgefunktionen kaldes fra et UI\footnote{User Interface} element i form af en slider. Denne kan ses på figur \ref{fig:slider}. 

\begin{figure}[H]
	\centering
	\includegraphics
	[width=140mm]{figures/slider.png}
	\caption{Slider til søgning filtreret efter afstand}
	\label{fig:slider}
\end{figure}

Actionen \textit{eager loader}\footnote{Ved eager loading forstås der at når man laver forespørgsel på en entitet, så loades der relateret entiter i samme forespørgsel} alle Barterads, og adresser på brugerne. Desuden loades brugerens egne koordinater. Derefter udregnes afstanden mellem brugeren selv og de resterende brugere. Alle der er inden for den valgte afstand på slideren bliver gemt i en liste. Derefter findes alle de gemte brugeres Barterads, der ikke allerede er byttet. Disse bliver returneret sammen med frontpage viewet, så resultatet af søgningen kan præsenteres. Denne funktionalitet giver mulighed for, at man kan finde de Barterads der ligger tæt på brugeren. Scenariet her er ligeledes beskrevet som "happy path", hvor situationer som fx. at brugeren ikke er logget ind ikke benævnes.


