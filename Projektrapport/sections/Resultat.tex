\chapter{Resultater}\label{ch:Resultat og Diskussion}
\section{Implementerede user stories}
Webapplikationen Bargainbarter er blevet udarbejdet efter prioriteterne beskrevet i MoSCoW-analysen. Tabel \ref{fig:Implementeringsstatus} viser alle user stories og status på deres implementation.

\begin{table}[H]
	\begin{tabular}{ | l | p{5cm} |}
		\hline
		\textbf{Userstory}  & \textbf{Status} \\ \hline
		US 1 - Oprette en bytteannonce  & Implementeret \\ \hline
		US 2 - Opret en brugerprofil & Implementeret \\ \hline
		US 3 - Se geografisk lokation for bytteannonce & Implementeret \\ \hline
		US 4 - Glemt password eller brugernavn & Implementeret \\ \hline
		US 5 - Kommentere en annonce & Implementeret \\ \hline
		US 6 - Søg efter bytteannonce & Implementeret \\ \hline
		US 7 - Se byttehistorik & Implementeret \\ \hline	
		US 8 - Kontakt en bruger direkte & Delvist implementeret \\ \hline	
		US 9 -Anmeldelse af en byttehandel & Implementeret \\ \hline	
	\end{tabular}
\caption{Implementeringstatus på de forskellige userstories}
\label{fig:Implementeringsstatus}
\end{table}

Som det ses i tabel \ref{fig:Implementeringsstatus} fremgår det, at alle userstories undtagen  US 8 er blevet fuldt implementeret. Man kan se resultatet af de udførte accepttest i bilagene \footnote{Se bilag - Dokumentation, sektion xx}.

\subsection{Chat-funktion}
Jævnfør US 8 var der ønske om, at kunne kontakte en anden bruger direkte vha. en chatfunktion\footnote{Se bilag - Dokumentation, sektion xx}. Det var ikke muligt at implementere denne feature fuldt ud. Men i stedet er der blevet implementeret et åbent chatrum, hvor alle sidens brugere kan kommunikere sammen i realtid i ét stort chatrum.\\ \\

\section{Teknologier og løsninger i systemet}
Dette afsnit beskriver hvilke teknologier og frameworks der er anvendt i de forskellige løsninger i systemet. Tabel \ref{fig:Webapplikation} beskriver de overordnede strukturelle løsninger i systemet, mens tabel \ref{fig:Features} beskriver løsninger på features i systemet.
\subsection{Webapplikation}

\begin{table}[H]
	\begin{tabular}{ | l | l |}
		\hline
		\textbf{Model i webapplikation}  & \textbf{Beskrivelse af løsning} \\ \hline
		Webapplikation framework  & ASP.Net MVC \\ \hline
		Database & Relationel database med Entity Framework \\ \hline
		Adgang til database & Repository pattern \\ \hline
		Styling & Bootstrap	 \\ \hline
	\end{tabular}
	\caption{Overblik over til webapplikationen}
	\label{fig:Webapplikation}
\end{table}

\subsection{Features}

\begin{table}[H]
	\begin{tabular}{ | l | p{5cm} |}
		\hline
		\textbf{Feature på webside}  & \textbf{Beskrivelse af løsning} \\ \hline
		Sortere/søge i annoncer ud fra geografisk lokation  & Google Maps API \\ \hline
		Rating med stjerner & Bootstrap Star Rating \\ \hline
		Chat-funktion & SignalR \\ \hline
		
	\end{tabular}
	\caption{Overblik over løsninger til features}
	\label{fig:Features}
\end{table}

