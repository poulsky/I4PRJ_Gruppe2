\chapter{Resultater}\label{ch:Resultat og Diskussion}
Webapplikationen Bargainbarter er blevet udarbejdet så brugeren kan oprette dem selv og bytteannoncer i systemet. Brugere kan chatte live med andre brugere af systemet. Bytteannoncer kan byttes med brugere, brugere kan kommenterer bytteannoncer og der kan søges efter bytteannoncer på baggrund af afstand eller kategori.

\begin{table}[H]
	\begin{tabular}{ | l | p{5cm} |}
		\hline
		\textbf{Userstory}  & \textbf{Status} \\ \hline
		US 1 - Oprette en bytteannonce  & Implementeret \\ \hline
		US 2 - Opret en brugerprofil & Implementeret \\ \hline
		US 3 - Se geografisk lokation for bytteannonce & Implementeret \\ \hline
		US 4 - Glemt password eller brugernavn & Implementeret \\ \hline
		US 5 - Kommentere en annonce & Implementeret \\ \hline
		US 6 - Søg efter bytteannonce & Implementeret \\ \hline
		US 7 - Se byttehistorik & Implementeret \\ \hline	
		US 8 - Kontakt en bruger direkte & Ikke implementeret \\ \hline	
		US 9 -Anmeldelse af en byttehandel & Implementeret \\ \hline	
	\end{tabular}
\caption{Implementeringstatus på de forskellige userstories}
\label{fig:Implementeringsstatus}
\end{table}

På tabel \ref{fig:Implementeringsstatus} fremgår det, hvilke userstories der er blevet implementeret i projektet. Som det fremgår af tabellen er alle userstories undtagen  US 9 blevet implementeret. Man kan se resultatet af de udførte accepttest i bilagene \footnote{Se bilag - Dokumentation, sektion xx}
\\ \\
\noindent Det betød, at følgende moduller er blevet udarbejdet i forbindelse med projektet:

\begin{itemize}
	\item Relationel database
	\item UnitOfWork og Generic Repository til kommunikation med databasen
	\item Chat 
	\item Søgefunktionalitet
	\item 
\end{itemize}