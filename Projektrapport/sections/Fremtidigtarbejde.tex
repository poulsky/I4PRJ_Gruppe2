\chapter{Fremtidig arbejde}
Dette afsnit beskriver hvordan der kunne arbejdes videre med BargainBarter.\\


\noindent En oplagt forbedring af projektet er, at få en større del af projektet til at køre client side som JavaScript. Det vil have den fordel, at projektet laver færre roundtrips til databasen og webserveren, hvilket ville give bedre performance.\\ 

\noindent Controllerne i projektet er med tiden vokset og er blevet uoverskuelige. Derfor ville omstrukturering af businesslogik være noget af det første i videreudviklingen af BargainBarter, så funktionerne bliver fordelt i klasser. Dette ville give et mere overskueligt og testbart design.\\

\noindent Hjemmesiden kan på nuværende tidspunkt ikke skalere perfekt på mobil og tablet. Da et af projektets hovedpunkter er brugervenlighed, vil optimering inden for dette område være ønskværdigt.\\

\noindent Algoritmer til at foreslå annoncer ville også være af stor nytte for brugeren, og vil derfor være prioriteret højt i forbindelse med videreudvikling af BargainBarter. \\

\noindent Når en bruger uploader et billede til en annonce, bliver billedfilen gemt i databasen. Det medfører at når man henter billeder fra databasen, vil systemet bruge unødvendige lang tid på at læse alle bytes. En løsning er at gemme billedet på serverens filsystem, og indsætte url'en til billedet i databasen. \\

\noindent Chatfunktionen blev kun delvist implementeret som et åbent chatrum, derfor ville denne funktionalitet være en naturlig del af fremtidig arbejde. \\

\noindent På BargainBarter er det kun muligt at tilgå brugere gennem deres annoncer. En optimering af søgefunktionen, så man kan søge direkte på brugere ville også være meningsfuldt.
 