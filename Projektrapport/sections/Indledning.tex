\chapter{Indledning}
Danskerne er i de seneste år blevet draget af deleøkonomi og genbrug. Tjenester som Uber, GoMore og Airbnb er i stor vækst, fordi de har gjort det nemt og let tilgængeligt for brugere at dele deres bil eller bolig og samtidig tjene lidt penge ved siden af. Genbrugsbutikker er mere populære end nogensinde, og folk er blevet mere bevidste omkring den miljøskadelige "køb og smid væk"-kultur.\cite{Genbrug} \\ \\ \noindent
Ideen til servicen BargainBarter udspringer af dette. Visionen er at gøre det nemt at bytte ting med andre folk i sit lokalområde. Gamle såvel som splinternye ting kan blive anvendt eller genanvendt i et andet hjem, fremfor at der bliver smidt ud og købt nyt. 
\\ \noindent

%Byttehandler er et overset og et uudforsket marked inden for genbrug.
%Det siges at et stykke tøj bliver i gennemsnit brugt seks gange, inden det bliver kasseret\cite{bytte}, så der er et behov for at folk kan forny deres garderobe uden pengepungen eller miljøet belastes. Desuden vil muligheden for at bytte fx. gamle computerspil, interiør eller køkkenudstyr også være prisværdigt. Men denne form for handel mangler en overskuelig platform til at servicere dette marked. \\ 

%Visionen bag dette projekt er, at lade brugerne

%Visionen bag dette projekt er derfor, at kunne organisere danskernes oplevelser med byttehandler på en overskuelig og brugervenlig måde. Tilmed vil byttehandler være valutaløse, hvilket skabe en tryghed for brugeren, da ens kortoplysninger ikke kræves.

\noindent BargainBarter er den service der forsøger at realisere denne vision. 
Målet med BargainBarter er en brugervenlig og valutaløs webapplikation, som lader brugere finde og bytte ting med andre i sit lokalområde eller hele landet. Fraværelsen af transaktioner med valuta gør henholdsvis, at flere ting kan blive genbrugt, og at BargainBarter skiller sig ud fra andre genbrugstjenester. \\ \\ \noindent
 %BargainBarter er opbygget som en Web App, der præsenterer brugeren for et interface der kan finde ting at bytte med i lokalområdet eller hele landet.
  Det skal være muligt at søge efter bytteannoncer filtreret efter afstand til brugeren, og man skal gennem en annonce kunne initiere en byttehandel med en anden bruger. Det skal være enkelt at registrere sig som bruger og man skal kunne kontakte andre brugere igennem en chat. Efter en afsluttet byttehandel, skal det være muligt for brugerne at give hinanden en vurdering, som er offentlig tilgængelig, så brugere har nemmere ved at vurdere troværdigheden af hinanden, og dermed beslutte om de vil indgå i en byttehandel med en given person. Webapplikationen skal kunne skalere til mobil, tablet og desktop, således at brugerne kan benytte systemet hvor som helst og når som helst.  \\ \\ \noindent
Med udgangspunkt i disse ønsker til systemet er der blevet opsat en række Gherkin's user stories\cite{Gherkin}, der beskriver hvad systemet skal kunne. Disse user stories ligger til grund for det overordnede system- og arkitekturdesign af BargainBarter, hvilket har ledt til det endelige design og implementering.  

\section{Ansvarsområder}
Tabellen \ref{fig:Ansvarstabel} viser hvem der er ansvarlig for de forskellige dele af projektet. Tabellen er vejledende, og viser kun hvem der har haft det overordnede ansvar for et område. De enkelte gruppemedlemmer har generelt været involveret i de fleste dele af projektet. Alle gruppemedlemmer har i øvrigt haft supportroller, som ikke vil omtales yderligere.  
\begin{table}[H]
	\begin{tabular}{ | l | p{5cm} |}
		\hline
		\textbf{Områder}  & \textbf{Ansvarlige} \\ \hline
		Præsentationslogik & Alle \\ \hline
		US 1 - Oprette en bytteannonce & Simon + Kasper \\ \hline
		US 2 - Opret en brugerprofil & Mikkel + Kasper \\ \hline
		US 3 - Se geografisk lokation for bytteannonce & Simon + Jeppe \\ \hline
		US 4 - Glemt password eller brugernavn & Mikkel \\ \hline
		US 5 - Kommentering en annonce & Lars \\ \hline
		US 6 - Søg efter bytteannoncer & Jeppe + Mikkel + Lars + Simon + Kasper \\ \hline
		US 7 - Se byttehistorik & Søren + Jeppe \\ \hline	
		US 8 - Kontakt en bruger direkte & Morten\\ \hline	
		US 9 - Anmeldelse af en byttehandel & Jeppe + Søren\\ \hline	
		Test & Lars + Simon + Mikkel  \\ \hline
		CI & Simon \\ \hline
		Systemarkitektur & Alle \\ \hline	
		Kravspecifikation & Alle \\ \hline	
	\end{tabular}
	\caption{Fordeling af ansvarsområder}
	\label{fig:Ansvarstabel}
\end{table}