\chapter{Indledning}
Danskerne er i de seneste år blevet draget af deleøkonomi og genbrug. Tjenester som Uber, GoMore og Airbnb er i stor vækst, fordi de har gjort det nemt og let tilgængeligt for brugere at dele deres bil eller bolig og samtidig tjene lidt penge ved siden af. \\
Byttehandler er dog et overset og et uudforsket marked inden for deleøkonomi. Det siges at et stykke tøj bliver i gennemsnit brugt seks gange, inden det bliver kasseret\cite{bytte}, så der er et behov for at folk kan forny deres garderobe uden pengepungen eller miljøet belastes. Desuden vil muligheden for at bytte fx. gamle computerspil, interiør eller køkkenudstyr også være prisværdigt. Men denne form for handel mangler en overskuelig platform til at servicere dette marked. \\ 
Visionen bag dette projekt er derfor, at kunne organisere danskernes oplevelser med byttehandler på en overskuelig og brugervenlig måde. Tilmed vil byttehandler være valutaløse, hvilket skabe en tryghed for brugeren, da ens kortoplysninger ikke kræves.  \\ \\
BargainBarter er den service der skal gøre det enkelt og nemt for brugere at bytte deres ting med andre brugere i deres lokalområde. BargainBarter er opbygget som en Web App, der præsenterer brugeren for et interface der kan finde ting at bytte med i lokalområdet eller hele landet. Det skal være muligt at søge efter specifikke emner og derved tilgå en annonce hvorfra man kan initiere en byttehandel med en anden bruger. Det skal være enkelt at registrere sig som bruger og det skal være enkelt at kunne kontakte andre bruger igennem en chat. Denne Web App kommer til at skalere perfekt til mobil, tablet og desktop, således at brugerne kan initiere en byttehandel hvor som helst og hvornår som helst.  \\ \\
Med udgangspunkt i disse overvejelser er der blevet opsat en række Gherkin's User Stories, der fortæller hvorledes systemet kommer til at virke. Disse User Stories vil lægge grund for et overordnet system- og arkitekturdesign af BargainBarter og senere hen, egentlig design af specifikke softwaremoduler.   