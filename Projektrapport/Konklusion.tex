\chapter{Konklusion}
Formålet med projektet var at udvikle et brugervenligt system, som hjælper brugere med at bytte ting med hinanden, for at lade gamle ting blive genbrugt, fremfor at bidrage til "køb og smid ud kulturen". Systemet blev implementeret som en webapplikation, som skulle kunne skalere til både mobil, tablet og desktop. Der var ønske om at lade brugeren søge efter annoncer filtreret efter afstand, og lade brugere kontakte hinanden via en chatfunktion. Efter en afsluttet byttehandel skulle det være muligt for brugere at give hinanden en vurdering. \\ \\ \noindent
Den endelige implementering er en ufuldstændig prototype af den fuldbyrdige vision af BargainBarter, og implementerer kun i nogen grad de ønskede features. Det er muligt at filtrere sin søgning efter afstand, og efter en afsluttet byttehandel brugere give hinanden en vurdering. Det er også muligt at kommunikere via chat, men den er ikke implementeret som kravene beskriver. Fremfor at være en privat chat mellem to brugere, fungerer den i skrivende stund som et åbent chatrum mellem alle brugere på siden. 
\\ \\ \noindent
For at få en brugevenlig webapplikation, blev der udarbejdet nogle ikke-funktionelle krav, som understøtter målet om høj brugervenlighe. Det indebærer blandt andet, at en bruger nemt kan navigere rundt på hjemmesiden med få klik, og at ventetiden på de forskellige funktionaliteter, ikke er for høj. \\ \\ \noindent
Den nuværende version af webapplikationen er delvist skalerbar. Siden kan godt anvendes på andre platforme, men dens layout er ikke optimalt at navigere i på eksempelvis en smartphone. Den måde hjemmesiden er bygget op tillader dog, at man med relativt begrænset arbejdsindsats kan gøre siden meget mere skalerbar. 



%Dette projekts hovedmål var at lave en brugervenlig platform, der gør det muligt for brugere at bytte med hinanden. Målet  var at gøre byttehandler valutaløse, så man f.eks. i stedet for at kassere sit tøj, kan bytte det til en ny garderobe. Derved er man fri for at tænke på prisen, og  kan fokusere på hvad man har brug for.

%For at få en brugevenlig web applikation, blev der stillet nogle krav i form af projektafgrænsning, MoSCoW analyse og ikke-funktionelle krav. Det indebærer blandt andet, at en bruger nemt kan navigere rundt på hjemmesiden og at ventetiden på de forskellige funktionaliteter, ikke er for høj. Her blev det blandt andet erfaret, at man med fordel kan gemme en URL til et billedet, i stedet for selve billedefilen, hvilket gør at man loader siden hurtigere. 

%Web applikationen opfylder "Must have" og "Should have" fra MoSCoW analysen. Disse funktionaliteter er alle tilgængelige inden for 1-4 klik, hvilket bakker op omkring en brugervenlig hjemmeside.

%Hjemmesiden er ikke skalerbar, da implementeringen af dette ikke var muligt inden for tidsrammen, derved er det ikke muligt for brugere at få en overskuelig oplevelse på mobil eller tablet.
