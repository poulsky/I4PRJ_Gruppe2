\chapter{Konklusion}
BargainBarter blev udviklet som en webapplikation, som lader brugere bytte ejendele med hinanden. Man kan oprette en brugerprofil, som tillader oprettelse af annoncer med ejendele som kan byttes. Det er muligt at søge efter annoncer på baggrund af forskellige søgekriterier, herunder kategori, titel og geografisk afstand. Efter en afsluttet byttehandel kan brugere give hinanden en rating, hvor de bedømmer oplevelsen af byttehandlen. I systemet er der implementeret otte ud af ni user stories, hvor den eneste, som ikke blev fuldt implementeret var chatfunktionen. Af kravene opstillet i MoSCoW-analysen er alle \textit{must have}, \textit{should have} og \textit{could have} implementeret.\\ 

\noindent Den nuværende version af webapplikationen er delvist skalerbar. Siden kan godt anvendes på andre platforme, men dens layout er ikke optimalt at navigere i på eksempelvis en smartphone. Den måde hjemmesiden er bygget op tillader dog, at man med relativt begrænset arbejdsindsats kan gøre siden meget mere skalerbar.\\

\noindent Gruppen har gjort sig en række væsentlige erfaringer omkring samspillet mellem design og testbarhed af et system. Desuden er der opnået erfaring indenfor webudvikling, databaser, og samarbejde under udvikling af et projekt.

 



%Dette projekts hovedmål var at lave en brugervenlig platform, der gør det muligt for brugere at bytte med hinanden. Målet  var at gøre byttehandler valutaløse, så man f.eks. i stedet for at kassere sit tøj, kan bytte det til en ny garderobe. Derved er man fri for at tænke på prisen, og  kan fokusere på hvad man har brug for.

%For at få en brugevenlig web applikation, blev der stillet nogle krav i form af projektafgrænsning, MoSCoW analyse og ikke-funktionelle krav. Det indebærer blandt andet, at en bruger nemt kan navigere rundt på hjemmesiden og at ventetiden på de forskellige funktionaliteter, ikke er for høj. Her blev det blandt andet erfaret, at man med fordel kan gemme en URL til et billedet, i stedet for selve billedefilen, hvilket gør at man loader siden hurtigere. 

%Web applikationen opfylder "Must have" og "Should have" fra MoSCoW analysen. Disse funktionaliteter er alle tilgængelige inden for 1-4 klik, hvilket bakker op omkring en brugervenlig hjemmeside.

%Hjemmesiden er ikke skalerbar, da implementeringen af dette ikke var muligt inden for tidsrammen, derved er det ikke muligt for brugere at få en overskuelig oplevelse på mobil eller tablet.
