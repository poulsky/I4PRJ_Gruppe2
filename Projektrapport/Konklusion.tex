\chapter{Konklusion}
Dette projekts hovedmål var at lave en brugervenlig platform, der gør det muligt for brugere at bytte med hinanden. Målet  var at gøre byttehandler valutaløse, så man f.eks. i stedet for at kassere sit tøj, kan bytte det til en ny garderobe. Derved er man fri for at tænke på prisen, og  kan fokusere på hvad man har brug for.

For at få en brugevenlig web applikation, blev der stillet nogle krav i form af projektafgrænsning, MoSCoW analyse og ikke-funktionelle krav. Det indebærer blandt andet, at en bruger nemt kan navigere rundt på hjemmesiden og at ventetiden på de forskellige funktionaliteter, ikke er for høj. Her blev det blandt andet erfaret, at man med fordel kan gemme en URL til et billedet, i stedet for selve billedefilen, hvilket gør at man loader siden hurtigere. 

Web applikationen opfylder "Must have" og "Should have" fra MoSCoW analysen. Disse funktionaliteter er alle tilgængelige inden for 1-4 klik, hvilket bakker op omkring en brugervenlig hjemmeside.

Hjemmesiden er ikke skalerbar, da implementeringen af dette ikke var muligt inden for tidsrammen, derved er det ikke muligt for brugere at få en overskuelig oplevelse på mobil eller tablet.
