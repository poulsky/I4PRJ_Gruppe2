\chapter{Accepttest}\label{ch:Accepttest}
I starten af projektet blev der beskrevet en række krav til systemets funktionelle krav og de ikke funktionelle krav. Disse krav kan ses i afsnit \ref{ch:Userstories}.

\section{Funktionelle krav}
De funktionelle krav er beskrevet ved brug af Gherkin. Accepttesten er derfor lavet på baggrund af Gherkin-scenariet. I det følgende afsnit, beskrives udførslen og resultat af accepttesten til de funktionelle krav, som er defineret til projeket.

\subsection{Opret En Bytteannonce}
{\color{blue}\textbf{SCENARIE:}} Oprette en bytteannonce \\ 
{\color{blue}\textbf{Når}}  bruger ønsker at oprette en bytteannonce i systemet \\
{\color{blue}\textbf{Så}} navigerer han til menupunktet ”Opret annonce” \\
{\color{blue}\textbf{Så}} udfylder han bytteannonce-skabelonen \\
{\color{blue}\textbf{Og}} trykker på ”Opret annonce”-knappen \\ \\
Det overnævnte scenarie blev kørt igennem og en bytteannonce blev oprettet tilhørende den pågældende bruger. \\
\textbf{Status - Godkendt}

\subsection{Opret En Brugerprofil}
{\color{blue}\textbf{SCENARIE:}} Opret en brugerprofil \\
{\color{blue}\textbf{Når}} bruger ønsker at oprette sig i systemet \\
{\color{blue}\textbf{Så}} navigerer han til Menupunktet ”Ny bruger” \\
{\color{blue}\textbf{Så}} indtaster han brugernavn, e-mail, password mm.\\
{\color{blue}\textbf{Så}} verificerer brugeren sig på tilknyttede e-mail addresse\\
{\color{blue}\textbf{Og}} brugeren kan nu logge ind med sin nye brugerprofil \\ \\
Det overnævnte scenarie blev kørt igennem og der blev oprettet en brugerprofil til brugere, der nu kan benytte systemet.  \\
\textbf{Status - Godkendt}

\subsection{Se Geografisk Lokation For Bytteannonce}
{\color{blue}\textbf{SCENARIE:}} Jakob vil se Gittes geografiske lokation \\
{\color{blue}\textbf{Når}} Jakob ønsker at se Gittes lokation \\
{\color{blue}\textbf{Så}} kigger han til højre for annoncebilledet \\
{\color{blue}\textbf{Og}} ser et kort, hvor gitte er markeret.
\\ \\
Det overnævnte scenarie blev kørt igennem og bytteannoncens geografiske lokation blev  præsenteret for brugeren.\\
\textbf{Status - Godkendt}

\subsection{Glemt Password Eller Brugernavn}
{\color{blue}\textbf{SCENARIE:}} Glemt password eller brugernavn \\
{\color{blue}\textbf{Når}} bruger ønsker at få tilsendt sit brugernavn med et nyt password \\
{\color{blue}\textbf{Så}} navigerer han til websidepunktet  “Glemt password eller brugernavn” \\
{\color{blue}\textbf{Så}} skriver brugeren sin e-mailadresse ind\\
{\color{blue}\textbf{Så}} finder brugeren den tilsendte e-mail \\
{\color{blue}\textbf{Og}} brugeren kan nu logge ind med det tilsendte password
\\ \\
Det overnævnte scenarie blev kørt igennem, og brugeren fik et nyt password og kan igen benytte systemet. \\
\textbf{Status - Godkendt}

\subsection{Kommenter En Annonce}
{\color{blue}\textbf{SCENARIE:}} Kommentere en annonce \\
{\color{blue}\textbf{Når}} bruger ønsker at kommentere en annonce
\\
{\color{blue}\textbf{Så}} skal bruger navigere ned til en kommentarboks under annoncen \\
{\color{blue}\textbf{Så}} skal bruger kunne skrive en kommentar i kommentarboksen\\
{\color{blue}\textbf{Så}} skal bruger trykke “send” ved siden af kommentarboksen \\
{\color{blue}\textbf{Og}} kommentaren vises da under kommentarboksen
\\ \\
Det overnævnte scenarie blev kørt igennem og der blev oprettet en kommentar til den pågældende bytteannonce. \\
\textbf{Status - Godkendt}

\subsection{Søg Efter Bytteannoncer}
{\color{blue}\textbf{SCENARIE:}} Søg efter bytteannoncer \\
{\color{blue}\textbf{Når}} bruger ønsker at søge efter bytteannoncer i systemet\\
{\color{blue}\textbf{Så}} navigerer brugeren til menupunktet ”Søg” \\
{\color{blue}\textbf{Så}} indtaster brugeren den varer som brugeren ønsker\\
{\color{blue}\textbf{Og}} trykker på “søg“-knappen \\
{\color{blue}\textbf{Og}} brugeren bliver præsenteret for de bytteannoncer, der opfylder søge-kriteriet. \\ \\
Det overnævnte scenarie blev kørt igennem og der blev søgt efer 'stol'. Brugeren blev præsenteret for en af annonce af en stol. \\
\textbf{Status - Godkendt}

\subsection{Se Byttehistorik}
{\color{blue}\textbf{SCENARIE:}} Se Byttehistorik – Webapplikationen \\
{\color{blue}\textbf{Når}} bruger ønsker at se sin byttehistorik i systemet\\
{\color{blue}\textbf{Så}} navigerer han til menupunktet ”Brugerprofil” \\
{\color{blue}\textbf{Og}} trykker på "Byttehistorik"-knappen \\ \\
Det overnævnte scenarie blev kørt igennem og brugeren kunne se sin byttehistorik. \\
\textbf{Status - Godkendt}

\subsection{Kontakt brugere direkte}
{\color{blue}\textbf{SCENARIE:}} Kontakt brugere direkte\\
{\color{blue}\textbf{Når}} Jakob ønsker at kontakte en bruger, Gitte, direkte \\
{\color{blue}\textbf{Så}} navigerer han til Gittes annonce  \\
{\color{blue}\textbf{Så}} navigerer han til knappen "Start chat" \\
{\color{blue}\textbf{Så}} trykker han på “Start chat" \\
{\color{blue}\textbf{Og}} begynder at chatte med Gitte \\ 
\\
Det overnævnte scenarie kunne ikke gennemføres, da der ikke er implementeret et privat-chat mellem to brugere. \\
\textbf{Status - Ikke godkendt}


\subsection{Anmeldelse af andre brugere}
{\color{blue}\textbf{SCENARIE:}} Anmeldelse af andre brugere \\
{\color{blue}\textbf{Når}} Jakob ønsker at vurdere byttehandlen med Gitte\\
{\color{blue}\textbf{Så}} navigerer han hen til “Afsluttede byttehandler”\\
{\color{blue}\textbf{Så}} trykker han “Vurder byttehandel” på den givne byttehandel med Gitte\\
{\color{blue}\textbf{Så}} angiver han en score mellem 1 og 5\\
{\color{blue}\textbf{Så}} skriver han en kommentar (valgfrit)\\
{\color{blue}\textbf{Så}} trykker han “Send”\\
{\color{blue}\textbf{Og}} hans anmeldelse vises nu på Gittes brugerprofil, og tælles med i Gittes samlede brugervurdering
\\ \\
Det overnævnte scenarie blev gennemført og Gitte fik en brugeranmeldelse af Jakob. Gittes brugervurdering gik fra 3 til 3.5. \\
\textbf{Status - Godkendt}

\section{Ikke funktionelle krav}
I følgende afsnit beskrives udførslen og resultatet af accepttesten for de ikke funktionelle krav.

\setlength{\arrayrulewidth}{0.3mm}
\setlength{\tabcolsep}{2pt}
\renewcommand{\arraystretch}{1.5}
\begin{table}[H]
	\begin{tabular}{ |p{1.4cm}|p{5.0cm}|p{5.0cm}|p{3.8cm}|p{1.0cm}| } 
		\hline
		\textbf{Krav} & \textbf{Beskrivelse} & \textbf{Forventet resultat} & \textbf{Resultat} & \textbf{\checkmark / X} \\
		\hline
		Punkt 1 & Der klikkes på "Annoncer"  & Brugeren kommer ind på en oversigt over alle annoncer & Brugeren kommer ind på en oversigt over alle annoncer ved blot at være på forsiden & \checkmark\\
		\hline
		Punkt 2 & Hjemmesiden bliver tilgået af 10 brugere samtidig & Der sker ikke nogen ændring i funktionaliteten for de enkelte brugere på hjemmesiden & Ingen ændring for de enkelte brugere af hjemmesiden & \checkmark \\
		\hline
		Punkt 3 & Der skrives ''Stol'' i søgefeltet og der trykkes på søg, når der trykkes på søg startes et stop-ur. Uret stoppes når Søgeresultaterne bliver vist på skærmen & Stop-uret viser mindre end 15 sekunder & Stop-uret viser 2.3 sekunder & \checkmark \\
		\hline
		%Punkt 4 & To brugere opretter en annonce samtidig på hver sin computer & Begge annoncer bliver oprettet uden nogen form for påvirkning af hinanden  &  & \\
		Punkt 4 & En computer tilkoblet et andet netværk end hjemmesidens server, går til hjemmesidens url-adresse (link til hjemmeside) via en Chrome Browseren & Browseren åbner forsiden til systemets hjemmeside  & Browseren skriver: 'Der kan ikke oprettes forbindelse til dette website' & X \\
		\hline
		Punkt 5 & Der søges efter annoncer inden for 10km & Den annonce, der ligger længst væk fra søgelokationen er mindre end 10100m & Den annonce, der er længst væk er 0.9km & \checkmark \\
		\hline 
		Punkt 6 & En bruger skriver ''Hej'' til en anden bruger gennnem det indbyggede chat-modul. Når der trykkes på send startes et stop-ur og uret stoppes, når den anden kan se beskeden & Stop-uret står på mindre end 10 sekunder & Testen er foretaget i den globale chat. Stop-uret viser 0.3 sekunder. & \checkmark \\
		\hline
		Punkt 7 & Skriv en brugervurdering på 500 tegn og prøver at skrive videre & Brugeren kan nu ikke tilføje flere tegn & Brugeren kan tilføje alle de tegn brugeren ønsker & X\\
		\hline
		Punkt 8 & Skriv en brugervurdering på under 500 tegn og tryk ''Send''. Når der trykkes på ''Send'' startes et stop-ur og uret stoppes, når bruger-anmeldelsen offentliggøres & Stop-uret viser mindre end 15 sekunder & Stopuret viser 1.5 sekunder & \checkmark \\
		\hline
		Punkt 9 & Tilgå hjemmesiden fra en af de afgivne enheder i afsnit \ref{ch:Ikkefunktionelle} via \url{http://10.29.0.30/bargainbarter}, log ind og opret en BarterAd  & En BarterAd er blevet opret i systemet & En BarterAd er blevet oprettet i systemet & \checkmark \\
		\hline
	\end{tabular}
	\caption{Accepttest af ikke-funktionelle krav del 1}
	\label{table:accepttest_udfort}
\end{table}