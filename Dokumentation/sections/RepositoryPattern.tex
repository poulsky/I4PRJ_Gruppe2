\section{Repository Pattern}
Repository pattern blev implementeret ved brug af forskellige dele og lag. I dette afsnit vil de forskellige dele blive beskrevet.
Hver  model blev tilknyttet et repository, der indeholdt de gængse CRUD operationer (Create, Read, Update, Delete) på databasen i forbindelse med hver model. 
Hvert repository er nedarvet fra et interface, dette er gjort på baggrund ad DIP, der gør at i forbindelse med test kan repositoriet mockes ud.
Dette repository bliver så tilgået gennem et Generisk repository som benytter sig af en template-baseret implementering af de generelle CRUD-oper
Et repository bliver tilgået i som et generisk repository. Et generisk repostory er en repository, der indeholder er en template baseret implementering af CRUD-operationerne. 
Af særlige interesante metoder i GenericRepository kan nævnet Get-metode, der nok også er den mest anvendte metode i forbindelse med DAL. Denne metode retunere en list af objekter af en given klasse.
Metoden:
\begin{verbatim}
 public virtual IEnumerable<TEntity> Get(Expression<Func<TEntity, bool>> 
 filter = null,Func<IQueryable<TEntity>, 
 IOrderedQueryable<TEntity>> orderBy = null,string includeProperties = "")
\end{verbatim}
Metoden indeholder en række af filtre, ekspression, orders og includeproperties, der gør, at man kan filtre og udvælge, hvilke data, der kommer tilbage fra get-metoden. Man kan bl.a. gennem brug af includeproperties komme rundt om EF Lazy-loading. 

GenericRepository's template implementering bliver kaldt igennem unitofwork-klassen. UnitOfWork-klassen er den eneste klasse, der kalder ned i databasen gennem GenericRepository. UnitOfWork-klassen indeholder af en af samtlige af de forskellige repositories. 

Dette er blevet valgt for at adskille buisness-logikken fra DAL.
