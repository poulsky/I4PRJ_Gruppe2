\section{Ansvarsbeskrivelse for controllers}

\subsection{Home}
Denne controller er ansvarlig for at vise de generelle ting for hjemmesiden. Dette vil sige hovedsiden, kontakt og en About section.  

Actions
\begin{itemize}
	\item Index
	Denne action returnere et view indeholdene alle barterads i databasen. 
	\item About
	Returnere det view der hedder about med information om den BargainBarter generelt.
	\item Contact
	Action der returnere et view med contact info omkring BB.
\end{itemize}


\subsection{BarterAd}
Denne controller er overodnet ansvarlig for, at brugeren kan oprette og opdatere Barteradds annoncer på hjemmesiden. Sammenfattet: CRUD operationerne på BarterAds.

Actions:
\begin{itemize}
	\item Index er siden, hvor den pågældende bruger får vist alle BarterAds og der en knap, der directer brugeren til MakeAd-actionen.
	\item MakeAd - er siden, hvor brugeren kan oprette en annonce. Det vil sige, at brugeren kan indtaste Annoncenavn, kategori, beskrivelse og uploade et billed til en annonce. I bunden
	\item EditAd - er siden, hvor brugeren kan opdatere en af sine bytteannoncer.
	\item 
	\item
\end{itemize}

\subsection{ShowBarterAds}
Denne controller er ansvarlig for at vise BarterAds fra databasen.

Actions 
\begin{itemize}
	\item Index skal vise alle BarterAds i omvendt kronologisk rækkefølge ift. alfabetisk rækkefølge. Der skal være mulighed for at brugeren kan vælge en kategori, der directer til ShowCategori-actionen.
	\item ShowCategory skal vise alle BarterAds, der er i den valgte kategori.
	\item EditAd - returnere siden, hvor brugeren kan opdatere en af sine bytteannoncer.
	\item 
	\item
\end{itemize}


\subsection{Search}
Search-controlleren er ansvarlig for søgning i annoncerne. 

Actions 
\begin{itemize}
	\item Index(bedre navn) skal tage en tekststreng og søge i annoncerne efter match og vise en side med de matchene annoncer.
\end{itemize}

\subsection{UserProfile}
Denne controller har til ansvar at styre ansvar omkring brugerprofiler. Dette er bl.a. at vise brugerprofiler og rette i brugerprofiler.

\begin{itemize}
	\item Index Redirector blot til forsiden. Dennegiver ikke mening, da der som regel skal være et brugerId for at vide hvilken brugerprofil det drejer sig om
	\item Edit (GET) hvis brugerprofilen er den samme som den bruger man tilgår siden med, får man lov at rette i profilen, ellers returneres blot et view hvor man kan se brugerprofilen
	\item EDIT (POST) De rettede oplysninger i brugerprofilen skrives ned i databasen. Brugerprofilen bliver herefter vist.
	\item ShowUserProfile sørger for at trække data udfra databasen og sende et pænt View til siden der viser de oplysninger en brugerprofil indeholder. Hvis profilen er den samme som den bruger der tilgår den, får brugeren mulighed for at rette i oplysninger vha. en redigerknap.
\end{itemize}
