\chapter{Arkitektur}

I projektet blev det valgt at lave hjemmesiden i ASP.NET MVC. I følgende afsnit redegøres for de systemets arkitektur og de anvendte patterns sammen med en begrundelse for deres valg.

\section{MVC}
Model-View-Controller er et GUI-pattern, der består af 3 dele:

Model: 

Controller:

View:

MVC blev valgt på baggrund af, at det er en integreret del af ASP.NET, der blev set som et spændende udfordring og framework af gruppen. 

\section{Repository Pattern}
Repository er et pattern, der indføres et ekstra lag (Repository-lag), der virker som DAL mellem buisness-logikken og databasen. Dette medfører at buisness-logikken ikke skriver direkte ned i databasen, så dette ikke bliver afhængig af databasen. Buisness-laget kalder bare ned i repository-laget, der så sørger for transaktionen med databasen. 
Repository pattern er blevet anvendt for at opnå en lavere kobling i systemet mellem databasen og buisness-logikken. Den lavere kobling gør systemet mere robust for ændre og gør samtidig også, at man kan teste på, at buisness-logikken der skal skrive ned i databasen. Da man kan stubbe/mocke repository-laget ud.

\section{Unit Of Work}
Unit of Work er et design pattern, der holder styr på in-memory opdateringer og skriver disse opdateringer til databasen. Dette er dermed den eneste klasse, der taler sammen med databasen.
Dette opnås ved at den vedlige holder en liste af alle objekter, der er bevet ændret i en database-transaction og sørge for at databasen bliver opdateret, når en transaction er færdig.