\section{Databasestruktur}
Der er i Barteradds projektet brug for at persistere en del data, og derfor brug for en eller anden form for database, det oplagte valg var at vælge en relationel database, da der er en masse struktur SQL serveren selv holder styr på.

\subsection{Persistent data}
Ud fra domæneanalysen er det klargjort hvad det er for noget data der er nødvendigt at persistere. Der er fundet yderligere data som det er ønsket at gemme i den løbende udvikling, men som ikke er opdateret i den daværende domæneanalyse.

Det valgte data er :
\begin{itemize}
	\item BytteAnnoncer
	\begin{itemize}
		\item Beskrivelse 
		\item Billede 
		\item Kategori
		\item Kommentarer
		\item Vurdering
	\end{itemize}
	\item BrugerProfil
	\begin{itemize}
		\item Brugernavn
		\item Password
		\item E-mail
		\item Lokation
		\item BytteHistorik
	\end{itemize}
\end{itemize} 

Det er vigtigt at understrege at der selvfølgelig er en række krav til disse data, der er defineret i systemarkitekturen. Brugeren skal ikke være tilgængelig for alle, og i forhold til almindelige kodeordsstandarder, skal passworded være hashet, således at applikationen ikke gemmer på nogen direkte version af kodeordet. 

\subsection{DAL} 
Databasen er i projektet lavet med entity framework, og den præcise forklaring af dette kommer i afsnit(INDSÆT!!!!). Men for selve strukturen i database overholder den ønskede systemarkitektur, hvor hvert lag ønskes både udskifteligt og testbart, er der lagt et repository pattern ned over selve databasen. Dette gør databaseteknologien yderst udskiftelig. Selve forklaringen af repository patternet kan findes i afsnit.(INSÆT!!!) Dette gør tests noget nemmere, da den resterende applikation kan testes med mocks af den unitofwork der er brugt, og er derved uafhængig af den konkrete database. Disse tests forklares i afsnit(Indsæt !!!!). Derved er Data acces laget lavet på en måde hvor det tilgås som interfaces hvor der nedenunder ligger den konkrete teknologi, hvor detaljerne i tilgangen til den relationelle database er gemt for applikationen.      

\subsection{Entity Framework}
Den konkrete teknologi der er valgt til at tilgå databasen er som nævnt entity framework. Dette er en ORM der tillader os at modellere databasen som var den objectorienteret. Frameworket er valgt da det er oplagt i sammenhæng med ASP.net, møder alle vores krav til en database og desuden er det den teknologi projektgruppen modtager undervisning i. Frameworket virker på en måde hvor den gennem de 10 database designregler opretter SQL tabeller med de tilsvarende fremmed nøgler. Derefter  opereres der på databasen via. LINQ, og selvfølgelig også stadig igennem vores repsitory pattern.      