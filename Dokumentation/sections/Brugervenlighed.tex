\section{Brugervenlighed}\label{sc:Brugervenlighed}
Brugervenlighed har været en afgørende faktor i forbindelse med udviklingen af systemet. Da BargainBarter er en webapplikation, skal den gerne appellerer til en bred målgruppe.
BargainBarter er forsøgt at lave brugervenligt ved at benytte frameworket: Bootstrap, så webapplikationen bliver skalerbar på tværs af forskellige enheder og skærmstørrelser. Når der nævnes brugervenlighed i forbindelse med BargainBarter, menes der især funktionalitet, og ikke så meget det æstetiske udtryk.\\
 Brugervenlighed er i høj grad blevet skabt igennem hjemmesidens design. I designet af hjemmesiden blev der tænkt over, at hjemmesiden skulle være let genkendelig og intuitiv at bruge. Konkret er det udført ved at bruge genkendelig elementer fra andre hjemmesider i designet. Disse elementer omfatter bl.a. navigationsbaren i toppen af siden. Mange elementer går igen på alle sider på hjemmesiden, så oplevelsen er ensartet. Navigationsbarens navne og links er blevet forfinet gennem projektet således, at de blev entydige og let forståelige. Et billede af den endelige navigationsbar kan ses på figur \ref{fig:navigationsbar}.

\begin{figure}[H]
	\centering
	\includegraphics
	[width=165mm]{figures/Navigationsbaren.png}
	\caption{Navigationsbaren for BargainBarter}
	\label{fig:navigationsbar}
\end{figure}

I projektet er der blevet lagt vægt på, at der skal kunne navigeres rundt på hele hjemmesiden på få klik.
Anvendelsen af ASP.NET MVC har gjort det simpelt at lave et layout, der går på tværs af de forskellige sider. Dette gøres ved et fælles layout der er blevet defineret i Shared.cshtml viewet.