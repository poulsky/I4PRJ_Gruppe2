\section{Brugervenlighed}\label{sc:Brugervenlighed}
Brugervenlighed har været en vigtig faktor i forbindelse med udviklingen af systemet. Da BargainBarter er en webapplikation er brugervenligheden vigtig for systemet,
da det skal appellerer til en bred målgruppe.
BargainBarter er forsøgt gjort brugervenligt ved at benytte frameworket: Bootstrap, så webapplikationen bliver skalerbar på tværs af forskellige enheder og skærmstørrelser.\\
Brugervenlighed er i høj grad også blevet skabt igennem hjemmesidens design. I designet af hjemmesiden blev der tænkt over, at hjemmesiden skulle være let genkendelig og intuitiv at bruge. Dette blev forsøgt ved at bruge genkendelig elementer fra andre hjemmesider i designet. Disse elementer omfatter bl.a. navigationsbaren i toppen af siden. Disse elementer går desuden igen på alle sider i hjemmesiden. Dette er gjort for, at brugeren altid kan bruge navigationsbaren til at navigere rundt på hjemmesiden. Navigationsbarens navne og links er også blev forfinet gennem projektet således, at de blev entydige og let forståelige. Et billede af den endelige navigationsbar kan ses på figur \ref{fig:navigationsbar}.

\begin{figure}[H]
	\centering
	\includegraphics
	[width=165mm]{figures/Navigationsbaren.png}
	\caption{Navigationsbaren for BargainBarter}
	\label{fig:navigationsbar}
\end{figure}

I projektet er der også blevet lagt vægt på, at der skal kunne navigeres rundt på hele hjemmesiden på få klik.
Anvendelsen af ASP.NET MVC har gjort det simpelt at lave et layout, der går på tværs af de forskellige sider. Da det fælles layout er blev defineret i Shared.cshtml viewet.