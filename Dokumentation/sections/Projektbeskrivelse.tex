\chapter{Indledning}\label{ch:Indledning}
Danskerne er i de seneste år blevet draget af deleøkonomi og genbrug. Tjenester som Uber, GoMore og Airbnb er i stor vækst, fordi de har gjort det nemt at være med på den nye trend. \\Byttehandler er en mulig del af denne trend, men denne form for handel mangler en overskuelig platform til at servicere dette marked. "Et stykke tøj bliver i gennemsnit brugt seks gange, inden det bliver kasseret"\footnote{https://www.information.dk/indland/2013/08/se-kjole-bytte}. Dette vækker et behov for at folk kan forny deres garderobe uden pengepungen eller miljøet belastes. Desuden vil muligheden for at bytte fx. gamle computerspil, interiør eller køkkenudstyr også være prisværdigt. \\
BargainBarter er platformen der skal gøre det nemt for brugere at bytte deres ting med andre brugere i deres lokalområde. BargainBarter er en webapplikation, hvilket muliggør brug på alle tænkelige devices og operativsystemer. \\ \\
Visionen for projektet er at organisere danskernes oplevelser ved byttehandler på en overskuelig og brugervenlig måde. 


\chapter{Projektbeskrivelse}\label{ch:Projektbeskrivelse}
Formålet med BargainBarter er at give danskerne et samlingssted for byttehandel. \\
Målet for dette system er at blive den største formidler af mulige byttehandler. Vi øjner en mulighed for at starte småt, og udvikle systemet større over tid, hvilket stemmer overens med at vi bruger agile udviklings metoder til at udvikle systemet. \\ Selve hjemmesiden vil blive skrevet i sproget XXXXXXX og kører på en XXXXXXX server client. Systemet er sat op til at bruge conteniues integration, så vi kan sikre kvalitet ved tests der bliver kørt automatisk. Det er desuden et mål for systemet at vi opnår stor brugervenlighed, så vi kan ramme et bredt publikum og ikke udelukker nogen baseret på, at siden er svær at navigere. Sidst men ikke mindst stiler vi efter at nå til et punkt hvor siden er hurtigt reagerende, og afvikler trafikken fra brugerne til vores servere samt databaser på en furniftig måde. 


