\chapter{Projektbeskrivelse}\label{ch:Projektbeskrivelse}
Danskerne er i de seneste år blevet draget af deleøkonomi og genbrug. Tjenester som Uber, gomore og Airbnb er i stor vækst, fordi de har gjort det nemt at være med på den nye trend. \\Byttehandler er en mulig del af denne trend, men denne form for handel mangler en overskuelig platform til at servicere dette marked. "Et stykke tøj bliver i gennemsnit brugt seks gange, inden det bliver kasseret"\footnote{https://www.information.dk/indland/2013/08/se-kjole-bytte}. Dette vækker et behov for at folk kan forny deres garderobe uden pengepungen eller miljøet belastes. Desuden vil muligheden for at bytte fx. gamle computerspil, interiør eller køkkenudstyr også være prisværdigt. \\
LocaTrade er platformen der skal gøre det nemt for brugere at bytte deres ting med andre brugere i deres lokalområde. LocaTrade er en webapplikation, hvilket muliggør brug på alle tænkelige devices og operativsystemer. \\ \\
Visionen for projektet er at organisere danskernes oplevelser ved byttehandler på en overskuelig og brugervenlig måde. 




