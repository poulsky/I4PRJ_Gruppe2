\chapter{Projektbeskrivelse}label{ch:Projektbeskrivelse}
Danskerne er i de seneste år gået mere og mere over i mod deleøkonomi og genbrug. Tjenester som Uber, gomore og DBA boomer, fordi de har gjort det nemt at være med på den nye trend. Byttehandlere er en mulig del af denne trend, men denne form for handel mangler en overskuelig platform til at servicere dette marked. Behovet for denne platform er stort i en tid, hvor: "Et stykke tøj i gennemsnit bliver brugt seks gange, inden det bliver kasseret" \footnote{https://www.information.dk/indland/2013/08/se-kjole-bytte}, er der behov for at folk kan forny deres garderobe eller dagligstue uden pengepungen eller miljøet belastes.
LocaTrade er platformen der skal gøre det nemt for brugere at bytte deres ting med andre brugere. LocaTrade er en webapplikation, der muliggøre brug på alle tænkelige devices og operativsystemer. 

Visionen for projektet er at forbedre danskernes oplevelser ved byttehandler. 




